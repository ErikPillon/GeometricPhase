% !TeX spellcheck = en_EN
% !TEX encoding = utf8
\documentclass[12pt,parskip=half, DIV=calc, BCOR=10mm, x11names]{scrbook}
\usepackage[utf8]{inputenc}
\usepackage[T1]{fontenc}
\usepackage[english]{babel}
\usepackage{amsmath,amsthm}
\usepackage{nameref}
\usepackage{wasysym}
\usepackage{cleveref}
\usepackage{url}
\usepackage{xargs}
\usepackage{lmodern, libertine}
\usepackage{xcolor, tcolorbox, empheq}
\usepackage{mathrsfs}
\usepackage{braket}
\usepackage{dsfont}
\usepackage{bbold}
\usepackage[ilines, headsepline]{scrpage2}
\setheadwidth[0pt]{textwithmarginpar}
\usepackage{caption}
\usepackage[round]{natbib}
\bibliographystyle{plainnat}
%\input{slike_1} % TikZ pictures
\clearscrheadfoot
\ihead{\headmark}
\ohead{\pagemark}

\setcounter{page}{3}

\addtokomafont{pagenumber}{\bfseries\Large\color{LightBlue4}}
\addtokomafont{pagehead}{\color{LightBlue4}}

\renewcommand{\textit}[1]{\textcolor{LightBlue4}{\emph{#1}}}
%\tcbuselibrary{skins,breakable}
\tcbuselibrary{theorems}
\tcbset{colback=blue!60!green!10!white,
	colframe=LightBlue4!50!black, ams nodisplayskip}

\usepackage{graphicx}
\graphicspath{{Images/}}

\usepackage{hyperref}

\usepackage{mathtools}
%Hide non-used labeled equations
%\mathtoolsset{showonlyrefs}

%Mark used and unused equations
\usepackage{refcheck} 
% ------------------------------------------------
% ------            ENVIRONMENTS        ----------
\theoremstyle{remark}
\newtheorem{rem}{Remark}
% ------------------------------------------------

% ------------------------------------------------
% ------             COMMANDS --------------------
\newcommand{\Hilbert}{\mathscr{H}} % Hilbert space
\newcommand{\Ham}{\mathcal{H}} % Hamiltonian
\newcommand{\diff}{\mathop{}\!d}

\newcommand{\ddt}{\frac{d}{dt}}
\newcommand{\qm}{quantum mechanics}
\newcommand{\wf}{wavefunction}
\newcommand{\Sch}{Schr\"odinger}
\newcommand{\R}{\mathbb{R}}
\newcommand{\C}{\mathbb{C}}
\newcommand{\pd}[2]{\frac{\partial #1}{\partial #2}}
\newcommand{\pds}[2]{\frac{\partial^2 #1}{\partial #2^2}}
\renewcommand{\div}{\mathrm{div}}
\newcommand{\dt}{\Delta t}
\DeclarePairedDelimiter\abs{\lvert}{\rvert}%
\DeclarePairedDelimiter\norm{\lVert}{\rVert}%
% ------------------------------------------------

% Book metadata
\title{Berry Phase: an Introduction}
\date{\today}
\author{Erik, Pillon}

\usepackage{emerald}
\begin{document}
\maketitle
\frontmatter

% All the lecture notes authored by Erik Pillon appearing on any webpage by Erik Pillon are licensed under a Creative Commons Attribuzione-Non commerciale-Non opere derivate 2.5 Italia License.

\section*{Acknowledgments}
This set of lecture notes accompanies prof. Mukundan Seminar course on Geometric Phase, taught from $ 10^{th} $ to $ 15^{th} $ March 2014.

The entire course is hosted on YouTube at the following address:
\begin{center}
	\url{https://www.youtube.com/playlist?list=PLhkiT_RYTEU15TB17zx32tbHGZ0mB6aSP}
\end{center}
These lecture notes are not endorsed by prof. Mukundan.

While I have tried to correct typos and errors made during the lectures (some helpfully pointed out by YouTube commenters), I have also taken the liberty to add and/or modify some of the material at various points in the notes. Any errors that may result from this are, of course, mine.

%If you have any comments regarding these notes, feel free to get in touch. Visit my
%blog for the most up to date version of these notes
%http://mathswithphysics.blogspot.com
%My gratitude goes to Dr. Schuller for lecturing this course and for making it available
%on YouTube.

\hfill {\ECFJD \Large Erik Pillon ~~~}
\tableofcontents

\mainmatter

%%%%%%%%%%%%%%%%%%%%%%%%%%%%%%%%%%%%%%%%%%%%%%%%%%%%%%%%%%%%%%%%%%%%%%%%%%%%
% INIZIO DISPENSA
%%%%%%%%%%%%%%%%%%%%%%%%%%%%%%%%%%%%%%%%%%%%%%%%%%%%%%%%%%%%%%%%%%%%%%%%%%%%


% !TeX encoding = UTF-8
% !TeX spellcheck = en_US

\chapter{Overview}

The work is given by a total amount of 6 sections; this first introductory section is an overview of the subject. The second section is a discussion about the Berry original work: first as a sempilified form and then as it was originally presented by prof. Berry. The third section will be the generalization given by Aharonov and A. and a year later by Samuel and A. In that third section we eill introduce also a brief mathematical interlude and some geometrical consideration about curvature. The fourth section is devoted essentially to the so called Kinematic Approach and we will bring the concept of Bargman invariance and its connection to curvature and the interesting appliation to the entire formalisma. Fifth section will be a second mathematical interlude on simplectic and riemannian manifolds. Sixth section will be about the null phase idea.


% !TeX encoding = UTF-8
% !TeX spellcheck = en_US

\chapter{Berry discovery of 1983-84}
Berry discovery of 1983-84\footnote{We say 1983-1984 since the original work was apparently submitted in 1983 but was at first instance rejected. However a preprint of his work must have been around since in 1983 another work on the Berry Phase appeared on Physical Review Letter. That's why sometimes the date can be misleading.} was a new discovery on the context about adiabatic of quantum mechanics and this work initiated a lot of work worldwide. In Berry's derivation several independent assumptions were made: initially this phase was called the Berry Phase for everyone but over time this name changed to geometric phase and as we'll see this concept is relevant also in classical wave-optical situations. It's quite remarkable that there are some chances that this concept can be used in condensed matter context. On the other hand we hope that the way we're presenting this work can naturally point out the applications.

As we said many work have been done to relax the assumptions that Berry done in order to defend Berry's work under more general conditions. The first important step was taken by \citet{aharonov1987phase}.
The second important by S and A was taken in 1988 and a third successful step was taken in 1993 and these are the successful and successive steps we'll describe. Apart from these improvements, people were also looking for earlier litteratures from different ideas much earlier then Berry. There are several of them but the most important we will touch upon is the on of Pancharatnam\citet{pancharatnam1956generalized} in 1957, that is, 27 earlier than Berry work. The fact that the work of Pancharatnam was in the direction of the geometric phase was pointed out by ... and ... in 1986. The other important early work relevant in this subject was the one of Bargman and Valentine in 1964 mostly 20 years before Berry working on discuss inf Wigner theorem of 1931, a theorem that Wigner had proved on how symmetric operators can be represented in quantum mechanics. So the th itself is very early (1931); many people had tried to give alternative proof of Wigner Theorem and one very important is given by Bargman in 1964, particularly elegant. The fact that Bargman work was important in the discovery of the Berry phase was pointed out and exploited by Syman and auth. in 1993

These lectures will describe all these thing and more mathematical relevant structures in a more chronological structures, but will not be strictly chronological rigorous. You'll find that many features of QM which we we might be familiar with they will be re examined, re-defined from the geometrical phase pov. When we will come to the kinematic approach we will define some applications. This should give an overview and an idea of the scope of these lectures.   
\section{Simplified Form}
We are now going to present the original Berry's work in a slightly simplified work. 

Having a quantum mechanical system in mind and a general setting, we will mainly deal with pure states $ \Hilbert $ with a time dependent Hamiltonian $ \Ham(t) $ governing the system and we have a state vector describing the system $ \psi(t) $.
The \wf must satisfy the (time dependent) \Sch equation 
\begin{equation}
i\hbar \ddt\psi(t)=\Ham(t)\psi(t)
\label{eq:2.1}
\end{equation}
If the Hamiltonian had be time independent, a formal solution of the \Sch equation is easy to find because what we have to do is to formally take the Hamiltonian and find out all its eigenfunctions and eigenvalues
\begin{equation}
\Ham \psi_n=E_n\psi_n, \quad n=1,2,\dots \qquad E_n \text{ real} 
\end{equation}
for simplicity let us assume everything is discrete while $ E_n $ are all real because of the hermiticity of the Hamiltonian and in general one has to express $ \psi $ as a linear combination of the basis elements and for each element has to add a time dependent exponential factor
\begin{equation}
\psi=\sum_n c_n\psi_n\to \psi(t)=\sum_n c_n e^{-iE_nt/\hbar}\psi_n.
\end{equation}
In principal this procedure is easy and really straightforward:
\begin{equation}
\Ham(t)\psi_n(t)=E_n\psi_n(t), \quad n=1,2,\dots \qquad E_n \text{ real}
\end{equation}
where we have implicitly supposed the $ E_n $ to be non degenerate and constant in time. $ \psi_n $ are called the stationary states of the system and of course the $ \psi_n $ form a complete set of orthonormal vector basis
\begin{align}
	\sum_n \Ket{\psi_n}\Bra{\psi_n}=\mathds{1}\\
	\left(\psi_n,\psi_k\right) = \delta_{nk}
\end{align}
but each $ \psi_j $ is defined up to an independent phase factor.

Let us now discuss the case of a time dependent Hamiltonian; at each time where have to use the Hamiltonian evaluated at that time, so generalizing what written above
\begin{equation}
\Ham(t) \psi(t)_n=E_n(t)\psi_n(t), \quad n=1,2,\dots \qquad E_n(t) \text{ real} 
\end{equation}
and as the Hamiltonian changes in time, then its eigenvalues do.
Of course at each time the eigenvalues form again a complete orthonormal set. This is, in principle, available to us.
\begin{rem}
	We stress again the fact that each eigenfunction is determined up to a phase. This factor can be both dependent on time and on $ n $.
\end{rem}
Now, there is no hope to recover the exact solution, even though you have solved the eigenvalue problem for each time. What we can do is to use eq.\eqref{eq:2.1} for rewriting $ \psi(t) $ as 
\begin{equation}
\psi(t)=\sum_n c_n(t)e^{-\frac{i}{\hbar}\int_{0}^{t}E_n(t')\diff t'}\psi_n(t)
\label{eq:2.4}
\end{equation}
and it will reduce to \eqref{eq:2.1} in the case of time independence.
So what we will get is
\begin{align}
i\hbar\sum_n\left(\dot{c_n}(t)\psi_n(t)-\frac{i}{\hbar}c_n(t)E_n(t)\psi_n(t)+c_n(t)\dot{\psi_n}(t) \right)e^{-\frac{i}{\hbar}\int_{0}^{t}E_n(t')\diff t'}\\
=c_n(t)E_n(t)\psi_n(t)e^{-\frac{i}{\hbar}\int_{0}^{t}E_n(t')\diff t'}
\end{align}
so, erasing the equal terms, we obtain
\begin{align}
\sum_n\left(\dot{c_n}(t)\psi_n(t)-\frac{i}{\hbar}c_n(t)E_n(t)\psi_n(t)+c_n(t)\dot{\psi_n}(t) \right)e^{-\frac{i}{\hbar}\int_{0}^{t}E_n(t')\diff t'}=0.
\label{eq:2.5}
\end{align}
and now we take the scalar product with the vector $ \psi_k(t) $
\begin{equation}
\dot{c}_n=-\sum_n c_n(t)e^{-\frac{i}{\hbar}\int_{0}^{t}(E_k(t')-E_n(t'))\diff t'} \left(\psi_k(t),\dot{\psi}_n(t) \right) \qquad \forall k 
\end{equation}
\begin{rem}
	The last equation is exact! There are no approximations involved so far!
\end{rem}

Let's focus for the moment on the term $ \left(\psi_k(t),\dot{\psi}_n(t) \right) $

\begin{equation}
\Ham(t) \psi(t)_n=E_n(t)\psi_n(t), \quad n=1,2,\dots  
\end{equation}
and we differentiate with respect to time on both sides
\begin{equation}
\pd{}{t}\Ham(t)\psi_n(t) +\Ham(t)\dot{\psi}_n(t)=\pd{}{t}E_n(t)\psi_n(t)+E_n(t)\dot{\psi}_n(t)
\end{equation}
and again we take again the scalar product with a generic state $ \psi_k(t) $
\begin{equation}
\left(\psi_k(t),\pd{\Ham(t)}{t}\psi_n(t) \right)+E_k(t)\left(\psi_k(t),\dot{\psi}_n(t) \right)=\dot{E}_n(t)\delta_{nk}+E_n\left(\psi_k(t),\dot{\psi}_n(t) \right)
\end{equation}
so it is appropriate to collect some terms, creating an energy difference, obtaining 
\begin{equation}
\left(E_n(t)-E_k(t)\right)\left(\psi_k(t),\dot{\psi}_n(t) \right)=-\dot{E}_n(t)\delta_{nk}+\left(\psi_k(t),\pd{\Ham(t)}{t}\psi_n(t) \right)
\end{equation}
and the previous result is true in general! There are no approximations involved at all!

If we restrict ourselves at the specific case $ k=n $
\begin{equation}
\dot{E}_(t)=\left(\psi_k(t),\pd{\Ham(t)}{t}\psi_n(t) \right)
\end{equation}
while in the case $ k\neq n $ we can straightforwardly derive from the eigenvalue problem \footnote{Remember that we assumed non degeneracy in energy spectrum in all the time}
\begin{equation}
\left(\psi_k(t),\dot{\psi}_n(t)\right)=\frac{\left(\psi_k(t),\pd{\Ham(t)}{t}\psi_n(t) \right)}{\left(E_n(t)-E_k(t)\right)}
\end{equation}
So we found an explicit expression relating the relative energy gap in time, the time derivative of the Hamiltonian and the scalar product of the time derivative of the wavefunction with any other wavefunction. We remark that each time dependent wavefunction $ \psi_n(t) $ is defined up to a phase factor that can depend on $ n $ and may depend on time. With this freedom we are left all alone. 

So from now on we agree to restrict the phase factor to be for each $ n $ such that 
\begin{equation}
\left(\psi_k(t),\dot{\psi}_n(t)\right)=0, \qquad \forall n
\label{eq:requirement}
\end{equation}
\begin{rem}
	Making use of the requirement \eqref{eq:requirement}, the phase freedom is eliminated. Once we chose a phase factor at time $ t=0 $ for $ \psi_n(0) $, no more flexibility is left.
\end{rem}

\begin{align}
\dot{c}_k(t)=&-\sum_{n\neq k}c_n(t)e^{i\int_0^t\omega_{kn}(t')\diff t'} \left(\psi_k(t),\dot{\psi}_n(t)\right) \\
=&\sum_{n\neq k} \frac{c_n(t)}{\hbar\omega_{nk}(t)}e^{i\int_0^t\omega_{kn}(t')\diff t'}\left(\psi_k(t),\pd{\Ham(t)}{t}\psi_n(t) \right) \qquad \forall k 
\label{eq:2.11}
\end{align}
where we defined $ \omega_{nk}(t)=\frac{E_k(t)-E_n(t)}{\hbar} $.
\begin{rem}
	We stress again that Eq.\eqref{eq:2.11} is obtain manipulating the \Sch equation only letting the Hamiltonian to vary in time.
\end{rem}

\subsection{Adiabatic condition}
Now we go to the \emph{adiabatic situation}. The so called \emph{Adiabatic condition} is a result mainly due to Born\&Fock in 1928 \cite{born1928m}

The main assumption we'll do is that the Hamiltonian we're considering is a slowly varying operator, that is to say
\begin{equation}
\pd{\Ham}{t}(t) \text{ is "small"}
\end{equation}
The term "small" will be quantitatively clear later on while, physically, it's reasonable to say that the quantities $ \psi_n(t), E_n(t) $ and $ c_n(t) $ are expected to slowly change in time.

Let's then continue with the original work of Fock and Born, back to 1928: suppose to have as initial solution $ \psi(0)=\psi_n(0) $, that is to say that the original initial state is equal to a particular eigenvector of the Hamiltonian at time $ t=0 $. So we have $ c_k(0)=\delta_{kn} $(the same chosen in \eqref{2.12})

Notice that our reasoning is perfectly consistent in the framework of the first order perturbation theory; since the term $ \pd{\Ham}{t} $ is explicitly appearing in \eqref{eq:2.11}, we can expand the equation and ignore all the upper degree terms, we have that for $ k\neq n $, 
\begin{equation}
\dot{c}_n(t)\simeq \dfrac{1}{\hbar\omega_{kn}}e^{i\omega_{kn}t}\left(\psi_k,\pd{\Ham}{t}(t)\psi_n\right)
\end{equation}
while for the $ n $-th term we have $ c_n(t)\simeq 1 $: it starts at 1 and stay fixed for all times. 
The $ \dot{c}_n(t) $ can be instead integrated and 
\begin{equation}
c_n(t)\simeq -\dfrac{1}{\hbar\omega_{kn}^2}\left(e^{i\omega_{kn}t}-1\right)\left(\psi_k,\pd{\Ham}{t}(t)\psi_n\right)
\end{equation}
and so, while the term $ c_n $ remains close to 1, all the other terms, despite the fact that they start from zero, they change significantly in the way explicited above. 

So,once is provided that the condition 
\begin{equation}
\frac{1}{\omega_{kn}}\abs{\left(\psi_k,\pd{\Ham}{t}(t)\psi_n\right)}\ll\hbar\omega_{kn}, \qquad \forall k\neq n
\label{eq:2.14}
\end{equation}
is satisfied, then
\begin{equation}
\psi(t)\simeq e^{-\frac{i}{\hbar}\int_{0}^{t'}E_n(t)\diff t}\psi_n(t)
\end{equation}
and this closes the statement of the adiabatic theorem of quantum mechanics.
\begin{rem}
	Eq.\eqref{eq:2.14} is the quantitative statement of what we mean by "adiabatic condition".
\end{rem}

And now comes the step taken by Berry; suppose $ \Ham(t) $ is cyclic, that is, $ \Ham(0)=\Ham(T) $ for some $ T $\footnote{Cyclic condition on the Hamiltonian}.
\begin{center}
	\textbf{Question: How behaves the approximate solution? Are they cyclic in a similar sense?}
\end{center}
Answer: the solution must be cyclic in some sense. Because of the non degeneracy and no crossing levels, we have that the eigenvalues of $ \Ham(T) $ are the same of $ \Ham(0) $
\begin{align}
E_n(T)=&E_n(0)\\
\psi(0)=\psi_n(0)\quad\Rightarrow& \quad\psi(T)\simeq e^{-\frac{i}{\hbar}\int_{0}^{T}E_n(t)\diff t}\psi_n(T)
\end{align}
by the Adiabatic Theorem. And now we ask: is this a cyclic solution? The answer is still \textbf{yes} but we have $ \psi(T)=\psi(0) $ apart from a phase, i.e., 
\begin{equation}
\psi(T)=\text{(n-dependent phse)}\psi_n(0)
\end{equation}
It follows then the following equalities:\footnote{They're all approximate in the sense of the adiabatic theorem, but we put the equality sign with no confusion}.
\begin{empheq}[box=\fbox]{align}
	\psi_n(T)&=e^{i\phi_{geom}^{(n)}}\psi_n(0)\label{eq:geometric_phase}\\
	\psi(T)&\simeq e^{i\phi_{tot}^{(n)}}\psi(0) \qquad i.e., \qquad \phi_{tot}^{(n)}=\mathrm{arg}\left(\psi(0),\psi(T)\right)\label{eq:total_phase}\\
	\phi_{tot}^{(n)}&=\phi_{geom}^{(n)}+\phi_{dyn}^{(n)} \qquad i.e. \qquad 
	\phi_{geom}^{(n)}=\phi_{tot}^{(n)}-\phi_{dyn}^{(n)}
	\label{eq:Berry_discovery}
\end{empheq}
where we defined 
\begin{equation}
\phi_{dyn}^{(n)}=\dfrac{i}{\hbar}\int_{0}^TE_n(t)\diff t
\end{equation}

So, in the end, this is the original work of Berry presented in a slightly different language.
\begin{rem}
	Eq. \eqref{eq:Berry_discovery} is regarded as the original discovery of Berry. Eq\eqref{eq:total_phase} is the definition of $ \phi_{tot} $; it is something we defined using the result of the Adiabatic Theorem: if there are no degeneracies andd no crossing levels, then every  approximate solution given by the adiabatic condition will also be cyclic. %instead the relation between the wavefunction at time $ T $ and time $ t=0 $ while 
	\eqref{eq:geometric_phase} instead defines how every phase of every wavefunction evolves in time provided that condition \eqref{eq:requirement} is  satisfied. We recall that \eqref{eq:requirement} is only a convention: given the general expression for $ (\psi_k,\dot{\psi}_{n}) $ in the particular case $ k=n $, there's no way for controlling the generic phase between those two, but if \eqref{eq:requirement} is satisfied, then a formula like \eqref{eq:geometric_phase} must exist. 
\end{rem}
\begin{rem}
	One could argue that changing the convention \eqref{eq:requirement} then the definition for \eqref{eq:geometric_phase} must change. This is absolutely reasonable, but nonetheless we will see that the geometric phase will not change under different assumptions: this invariance is one of the most important properties of the Berry Phase. 
\end{rem}

% ---------------------------------------------
\section{Berry original derivation with parameter space}
We want no to present the work performed by Berry in his original work in 1983 in the spirit of the parameter space.

The original assumption was that the Hamiltonian, apart from being hermitian, depends on a set of external parameters, let's say $ \rp $, that is $ \Ham\equiv\Ham(\rp) $ with $ \rp $ belonging to a multidimensional real parameter space. Then Berry imagined the following situation: suppose the external classical environment is slowly changing in time, like when we put a spin $ \frac{1}{2} $ particle in a magnetic field and we slowly change the external magnetic field. Imagine then the real parameter space being slowly dependent in time, that is to say, $ \rp $ itself to be time dependent, so that the Hamiltonian si explicitly time dependent $ \Ham(\rp(t)) $. The parameter now that varies adiabatically is of course now $ \rp(t) $.
\begin{rem}
	We recall that $ \rp(t) $ is a set of real independent parameters and thanks to the Adiabatic condition the quantity $ \pd{\rp}{t} $ is small.
\end{rem}

\begin{equation}
\mathcal{C}=\left\{\rp(t)\mid 0\leq t\leq T \right\}
\label{eq:Curve_C}
\end{equation}
\textbf{Figure missing! Place here as soon as possible!}

The curve $ \mathcal{C} $ is the curve traced by $ \rp(t) $ letting $ t $ varying. $ \mathcal{C} $ is of course cyclic.
The curve $ \mathcal{C} $ is then our domain of interest and for each point in $ \mathcal{C} $ we have an associated Hamiltonian.
\begin{align}
\Ham(t)\Ket{n;\rp}=E_n(\rp)\Ket{n;\rp}\\
\Braket{n';\rp|n;\rp}=\delta_{n'n}
\end{align}
and we have of course an orthonormal basis at each point of the multidimensional parameters space.
\begin{rem}
	To be precise, we should refer to the eigenvalues $ E_{n}(\rp) $ not as eigenvalues of the Hamiltonian but as \emph{eigenvalues of the Hamiltonian at a certain point $ \rp $ of the parameter space}. We recall moreover that each wavefunction $ \Ket{n;\rp} $ is defined up to a phase factor that may depend on $ n $ and on $ \rp. $ 
\end{rem}
\begin{rem}
	Thanks to the non degeneracy of the eigenvalues of the Hamiltonian, we have no ambiguity in defining the eigenvectors of the Hamiltonian, that is to say, there are not crossing levels of energy. Let's say for example that we start from the point $ \rp(0) $ in tha parameter space, then we can recover a set of well defined single valued wavefunctions for the Hamiltonian $ \Ham(\rp(0)) $. Letting now $ t $ varying from $ 0 $ to $ T $ we have a complete set of well defined single valued vawefunctions for every point and when $ t=T $ we have the same set of eigenfunctions as we had in $ t=0. $ Each eigenfucntion remains, anyway, defined up to an arbitrary phase. 
\end{rem}

At this stage Berry recover the Adiabatic Theorem and tries to solve the classical \Sch equation in the case in which the initial state is the $ n $-th eigenstate of the Hamiltonian at $ t=0 $:
\begin{equation}
\begin{cases}
i\hbar\dot{\psi}(t)=\Ham(\rp(t))\psi(t)\\
\psi(0)=\Ket{0;\rp(0)}
\end{cases}
\end{equation}
which, for intermediate times, within the validity of the Adiabatic Theorem, gives the solution
\begin{equation}
\psi(t)\simeq e^{-\frac{i}{\hbar}\int_{0}^{t}E_n(\rp(t'))\diff t'+\gamma_n(t)}\Ket{n;\rp(t)}
\label{eq:gen_solution}
\end{equation}
and we easily see that $ \gamma_n(0)=0, $ since it has to match with the initial condition.

The first term in the exponential of eq.\eqref{eq:gen_solution} is the \emph{dynamical phase} while the term $ \gamma_n(t) $ is the geometric phase. This latter phase is non integrable and so far we are not ready to handle it.

So the next step to take is to take the (general) solution \eqref{eq:gen_solution}, to plug it into the \Sch equation itself and try to extract an equation for $ \gamma_n(t) $.
%% --------- Second Lecture ------------
%From the initial condition $ \psi(0)=\psi_n(0) $ we have
%\begin{equation}
%\psi(t)=\sum_{n'}c_{n'}(t)e^{-\frac{i}{\hbar}\int_{0}^{t}E_n(t')\diff t'}\psi_{n'}(t)
%\end{equation}
%
%Then, if $ n'\neq n $, $ \dot{c}_n(t)\simeq 0 $ if $ \abs{\left(\psi_{n'},\pd{\Ham}{t}(t)\psi_n \right) }<<(E_{n'}-E_{n})/\hbar, $
%then we have $ \dot{c}_n(t)\simeq0, c_n(t)\simeq 1. $ And so this means that the \Sch solution takes the form of:
%\begin{equation}
%\psi(t)\simeq e^{-\frac{i}{\hbar}\int_{0}^{t}E_n(t')\diff t'}\psi_{n'}(t) \xrightarrow[t\to T]{} e^{-\frac{i}{\hbar}\int_{0}^{T}E_n(t)\diff t}\psi_{n}(t)\underbrace{\psi_n(T)}_{e^{i\phi_{geom}^{(n)}}\psi_n(0)}
%\end{equation}
%
%\begin{equation}
%\psi(T)\simeq \underbrace{e^{i\phi_{tot}^{(n)}}}_{\underbrace{e^{\frac{i}{\hbar}\int_{0}^{T}E_n(t)\diff t}}_{\text{dyn}}\times\underbrace{e^{i\phi_{geom}^{(n)}}}_{\text{geom}}}\psi(0).
%\end{equation}
%
\begin{rem}
	The set $ \Ket{n;\rp} $ form a set of single-valued eigenvectors in the parameter space. Pay attention that this condition is the crucial difference in spirit with the derivation given in the previous section. There the original assumption was to put $ \left(\psi_n(t),\dot{\psi}_n(t) \right) $ in order to eliminate any arbitrary freedom on the choice of the initial wavefunction.
\end{rem}

%Suppose to have the following situation: $ \psi(0)=\Ket{n,\rp(0)} $, that is, the initial condition is in one of the eigenstates for the Hamiltonian.  Its time evolution is then defined by
%\begin{equation}
%\begin{cases}
%\psi(t)\simeq e^{-\frac{i}{\hbar}\int_{0}^{t}E_n(\rp(t'))\diff t'+i\gamma_n(t)}\Ket{n;\rp(t)}\\
%\gamma_n(t)=0
%\end{cases}
%\end{equation}

\begin{rem}
	What Berry did is the following: he applied the Adiabatic Theorem and obtained an approximated solution for $ \psi(t) $. He then put $ \psi(t) $ itself into the \Sch Equation and derived an equation of motion for $ \gamma_n(t) $
\end{rem}

So, putting eq.\eqref{eq:gen_solution} into the \Sch equation, is equivalent to write (as a consequence of the Adiabatic Theorem)
\begin{equation}
\dot{\gamma}_n(t)\Ket{n;\rp(t)}\simeq i\ddt \Ket{n,\rp(t)}
\end{equation}
and then sandwiching with a Ket,
\begin{align}
	\dot{\gamma}_n(t)&\simeq i\Bra{n;\rp(t)}\ddt\Ket{n:\rp(t)}\\
	&=i\Bra{n;\rp(t)}\underline{\nabla}\Ket{n;\rp(t)}\dot{\rp}(t)\footnote{We have to stress that by $ \underline{\nabla} $ we mean the "parameter derivative", that is the derivative with respect to the component of the vector $ \rp $}
\end{align}

\begin{rem}
	We notice that the result \begin{equation}
	\dot{\gamma}_n(t)\simeq i\Bra{n;\rp(t)}\ddt\Ket{n:\rp(t)}
	\end{equation}
	is a consequence of the \Sch Equation and not a condition imposed on the system.
\end{rem}
\begin{rem}
	We have $ \Ket{n;\rp(0)}=\Ket{n;\rp(T)} $ since they're globally well defined.
\end{rem}

Performing the circuitation along the curve $ \mathcal{C} $ defined in \eqref{eq:Curve_C}, we have
\begin{equation}
\gamma_n(T)\equiv \gamma_n(0)=i\oint_{\mathcal{C}}\Bra{n;\rp}\underline{\nabla}\Ket{n;\rp}\dot{\rp}(t)\diff t=i\oint_{\mathcal{C}}\Bra{n;\rp}\underline{\nabla}\Ket{n;\rp}\cdot\diff\rp
\label{eq:2.26}\end{equation} 
and this is exactly what we wrote above.

In the end, we see that 
\begin{equation}
\Bra{n;\rp}\underline{\nabla}\Ket{n;\rp}=i\Im\Bra{n;\rp}\underline{\nabla}\Ket{n;\rp}
\label{eq:2.27}
\end{equation}
and so plugging \eqref{eq:2.27} into \eqref{eq:2.26} we obtain
\begin{align}
\gamma_n(\mathcal{C})&=-\Im \oint \Bra{n;\rp}\underline{\nabla}\Ket{n;\rp}\cdot\diff\rp\\
&=-\Im\iint_{\mathcal{S}}\nabla\wedge\Bra{n;\rp}\underline{\nabla}\Ket{n;\rp}\cdot\diff\mathcal{S}\text{\footnotemark}\\
&=-\Im\iint_{\mathcal{S}}\left(\underline{\nabla}\Bra{n;\rp}\right)\wedge\left(\underline{\nabla}\Ket{n;\rp}\right)\cdot\diff\mathcal{S}\\
&= -\Im \iint_{\mathcal{S}}\left(\sum_{m\neq n}\underline{\nabla}\Bra{n;\rp}\right)\Ket{m;\rp}\wedge\Bra{m;\rp}\underline{\nabla}\Ket{n;\rp}
\end{align}
\footnotetext{Following Berry's argument, we are now restricting ourselves in the case in which the parameter space is a 3D space, so the simplest form of Stoke's Theorem can be applied.}
and the term $ m=n $ is neglected since gives a zero contribution: a pure imaginary term times a pure imaginary term gives an only real contribution, so neglected by the operator $ \Im. $

Then from $ \Ham(\rp)\Ket{n;\rp}=E_n(\rp)\Ket{n;\rp} $, applying the gradient operator on both sides, we obtain for $ m\neq n $
\begin{equation}
\Bra{m;\rp}\underline{\nabla}\Ket{n;\rp}=\dfrac{\Bra{m;\rp}\underline{\nabla}\Ham(\rp)\Ket{n;\rp}}{E_n(\rp)-E_m(\rp)}.
\end{equation}
The final result is then
\begin{equation}
\gamma_n(\mathcal{C})=-\iint_{\mathcal{S}}V_n(\rp)\cdot\diff\mathcal{S}
\end{equation}
where we defined $ V_n $ as 
\begin{equation}
V_n(\rp):=\Im\sum_{m\neq n}\dfrac{\Bra{n;\rp}\underline{\nabla}\Ham(\rp)\Ket{m;\rp}\wedge\Bra{m;\rp}\underline{\nabla}\Ham(\rp)\Ket{n;\rp}}{\left(E_n(\rp)-E_m(\rp)\right)^2}
\label{eq:2.31}
\end{equation}
\begin{rem}
	Berry's comment at this stage is that the result does not depend on $ \rp $ since the gradient is only acting on $ \Ham(\rp) $, so it is independent on the choice of the phase of the wavefunction. The final take at home message is then that the geometric phase $ \gamma_n(\mathcal{C}) $ does not depend on the particular choice of the phase of the eigenvector $ \Ket{n;\rp} $ of the Hamiltonian at each point of the parameter space, under the assumption that \emph{the eigenvectors are globally well defined}.
	\label{rem:freedom}
\end{rem}
Remark \ref{rem:freedom} leads to an enormous amount of freedom in the choice of the global phase $ \chi(\rp) $
\begin{equation}
\Ket{n;\rp}\to e^{i\chi(\rp)}\Ket{n;\rp}
\end{equation}
\subsection{Two fold degeneracies}
Let us suppose that we are in the proximity of a two fold degeneracy. By degeneracy we mean that two levels of the Hamiltonian happen to be degenerate of a certain point of the phase space.

%What is the phase in the proximity of a two folds degeneracy?

We can suppose in $ \rp=0 $ we have a degeneracy for the Hamiltonian. So if we are far from zero, we have o problems at all, but if we are close to $ 0, $ we have to work in a different way. The two crossing levels are the most important, so from a Quantum Mechanical point of view, the problem is only a two level problem.
\begin{align}
	\Ham(\underline{0})=0; \qquad & \Ham(\rp)=\Ket{\pm;\rp}=E_{\pm}(\rp)=\Ket{\pm;\rp}\\
	& E_{+}(\rp)>E_-(\rp),\qquad \rp\neq 0\\
	& E_{+}(\underline{0})=E_{-}(\underline{0})=0.
\end{align} 
So, now, without any loss of generality, we can make use of the Pauli Matrices for defining the Hamiltonian to be in the following way
\begin{align}
	\Ham(\rp)=\dfrac{1}{2};\qquad E_{\pm}(\rp)=\pm\dfrac{1}{2}R, \qquad R=\abs{\rp}\\
	\Ket{+;\rp}\Bra{+;\rp}=\dfrac{1}{2}\left(1\pm \hat{R}\cdot\underline{\sigma} \right)\label{eq:2.33}
\end{align}
And so we have to plug \eqref{eq:2.33} into \eqref{eq:2.31}: the summation becomes an only single term and, recognizing that $ E_n(0)-E_m(0)=+\frac{1}{2}R-(-\frac{1}{2}R) $, we obtain 
\begin{equation}
\underline{V}_{+}(\rp)=\frac{R_j}{2R^3}
\label{eq:2.35}
\end{equation}

So now comes the result of Berry: the phase gained by the \wf is exactly given by the integration of \eqref{eq:2.35} and is the solid angle shifted by the curve $ \mathcal{C} $. Berry refereed to this result as the \textit{magnetic field of a "magnetic monopole"}
\begin{equation}
\gamma_{+}(\mathcal{C})=-\dfrac{1}{2}\Omega\left[\mathcal{C}\right], \qquad \Omega\left[\mathcal{C}\right]=\text{solid angle at \underline{0}}
\end{equation}
% !TeX encoding = UTF-8
% !TeX spellcheck = en_US
\chapter{Aharonov-Anandan and Samuel-Bhandari generalizations}
\section{Aharonov-Anandan generalization (1987)}
The first relaxation was taken in \cite{aharonov1987phase} where the authors considered the Adiabatic Hypothesis: they showed that the latter is not needed in order to get a geometric effect. 

Let us take a $ t $-dependent Hamiltonian
\begin{equation}
i\hbar\ddt\psi(t)=\Ham(t)\psi(t)
\end{equation}
and assume that we found a solution for the problem that is cyclic in the sense given in the previous chapter, that is,
\begin{equation}
\psi(T)=e^{i\psi_{tot}}\psi(0)
\label{eq:3.1}
\end{equation}
No assumptions on the cyclicity of the Hamiltonian is assumed in \eqref{eq:3.1}, nor no assumption on the adiabaticity of the Hamiltonian has been made neither and we also don't need the fact that $ \Ham(T)=\Ham(0) $, but if we have a periodicity on the solution we can associate to the \wf a new concept of geometric phase\footnote{from now on we'll distinguish the Berry phase from the present geometric phase since the concept we're dealing right now is quite different form the initial idea given by Berry: the concept of cyclicity and adiabaticity are now not needed.}: 
\begin{equation}
\phi_{geom}=\phi_{tot}-\phi_{dyn}
\end{equation}
where
\begin{equation}
\phi_{dyn}=-\dfrac{i}{\hbar}\int_{0}^{T}\Bra{\psi(t)}\Ham(t)\Ket{\psi(t)}\diff t
\label{eq:phi_dyn}
\end{equation}
Before exploit extensively the result achieved by Aharonov and Anandan, we can summarize them briefly:
\begin{enumerate}
	\item \textbf{Clear definition of geometric phase}: under the condition of cyclicity of the solution $ \psi(t) $, it is possible to derive the quantity $ \phi_{tot} $. Then, given the prescription for the definition of a dynamical phase, Eq.~\eqref{eq:phi_dyn}, is possible to derive the geometric phase understood as a difference between two terms.
	In particular, in the case of an adiabatic condition for the Hamiltonian, the calculation reduces to the ones already performed by Berry.
	\item \textbf{The geometric phase depends only on the projection on the ray space}: it is not anymore important the \wf itself but it is only important the movement of the trace of the solution projected on the space of density matrices. Being a pure state, well defined, single valued unitary vector in Hilbert space for each time, $ \psi(t) $ can be uniquely determined by the corresponding 1 dimensional projection operator, or, as we will call it, \emph{a point in the ray space}. In a certain sense, Berry already stated this result, but the A-A were able to state this result in a more general perspective, stating that the geometric phase lives in the ray space.
	\item \textbf{The parameter space concept is not needed to arrive to tha concept og geometric phase}: we're not required to know the Hamiltonian acquire its dependence on the \rp vector. 
\end{enumerate}

% --------------------------------------------------
\section{First Mathematical Interlude}
So far, we've been concerning an Hilbert space $ \Hilbert $ of dimension $ \dim\Hilbert=N $ (where $ N $ could be eventually infinite) and where all the possible states are defined by $ \psi $. We define now the \textit{unit sphere} of this space, defined by all vectors that have unit norm:
\begin{equation}
\B=\left\{\psi\in\Hilbert\colon \Braket{\psi,\psi}=1\right\}\subset\Hilbert
\end{equation}
We notice immediately that $ \B $ is not a linear vector space and that its dimension is $ N-1 $. 

The group $ U(1) $ of phase factors act on this set in an obvious way:
\begin{equation}
\psi\in\B\Rightarrow \psi'=e^{i\alpha}\psi\in\B,\qquad 0\leq\alpha\leq 2\pi
\end{equation} 

We now want to introduce the concept of \textit{ray space}, that we will denote by $ \mathcal{R} $, as the quotient of $ \B $ with the action of $ U(1) $. 
The idea is that two vectors that differ only by a phase factor should be regarded as equivalent.
We can also define now the equivalence class by fixing $ \psi $ and letting $ \alpha $ varying:
\begin{equation}
\mathcal{R}:=\left\{\rho(\psi)=\Ket{\psi}\Bra{\psi} \text{ or } \psi^{\dagger}\psi \mid \psi\in\B \right\}.
\end{equation}
We denoted by $ \rho $ the well known projective operator, while we have to take care that the above set is not a linear vector space. 
We notice that if we denote by $ N $ the dimension of the Hilbert space $ \Hilbert $, then we have that $ \dim(\mathcal{R}) $ has $ 2(N-1) $ real dimensions (or equivalently $ N-1 $ complex dimensions), or, in mathematical formalism, the space $ \cp{N-1} $.
We define as well the inverse of the projection operator, the operator $ \pi $, in the following way:
\begin{align}
\pi\colon \B&\to\mathcal{R} \\
\psi&\rightarrow \rho(\psi)=\psi^{\dagger}\psi\in\mathcal{R}.
\end{align}

Points in the real space are in $ 1-1 $ correspondence with the pure real state of the system. The mathematical description is
\begin{equation}
\rho(\psi)\in\mathcal{R}\to\pi^{-1}(\rho(\psi)):=\left\{\psi'=e^{i\alpha}\psi\in\B\mid\psi \text{ fixed }, 0\leq\alpha\leq 2\pi \right\}\subset \B
\end{equation}

We now consider curves on the space $ \B $. Take a curve $ \mathcal{C} $ 
\begin{equation}
\mathcal{C}:=\left\{\psi(s)\in\B |s_1\leq s\leq s_2 \right\}\subset \B
\end{equation}
with suitable smoothing conditions on it that will depend on the use we will make of this curve. We consider its projection into the real space $ \mathcal{R} $ through $ \pi $ into the set 
\begin{equation}
C=\pi\left[\mathcal{C}\right]=\left\{\rho(\psi(s))=\psi(s)\psi^{\dagger}(s)\in\mathcal{R}|s_1\leq s\leq s_2 \right\}\subset\mathcal{R}.
\end{equation}
So for our purposes the most general lift will be 
\begin{equation}
\mathcal{C}'=\left\{\psi'(s)=e^{i\alpha(s)}\psi(s)|\psi(s)\in\mathcal{C}, s_1\leq s\leq s_2 \right\}\subset\B,
\end{equation}
that denotes all the possible lifts of the original projected curve. 

Now, given a general curve $ \mathcal{C} $, comes natural to define the tangent vector 
\begin{align}
	u(s)&=\dfrac{d}{ds}\psi(s)=\dot{\psi}(s)\label{eq:3.11}\\
	(\psi(s),\dot{\psi}(s))&=\Im(\psi(s),u(s))\label{eq:3.12}\\
	u'(s)&=e^{i\alpha(s)}(u(s)+i\dot{\alpha}(s)\psi(s))\\
	u_{\perp}(s)&=u(s)-\psi(s)(\psi(s),u(s))=u_{\perp}'(s)=e^{i\alpha(s)}u_{\perp}(s)\\
\end{align}
where it easy to show that Eq.~\eqref{eq:3.12} is a trivial consequence of Eq.~\eqref{eq:3.11}. The general length of a curve can be evaluated explicitly as
\begin{align}
	\mathcal{L}\left[\mathcal{C}\right]&=\int_{s_1}^{s_2}\left(u_{\perp}(s),u_{\perp}(s)\right)^{1/2}\cdot\diff s \label{eq:3.15}\\
	&=\int_{s_1}^{s_2}\left\{\left(\dot{\psi}(s),\dot{\psi}(s)\right)-\left(\psi(s),\dot{\psi}(s)\right)-\left(\dot{\psi}(s),\psi(s)\right) \right\}^{1/2} \diff s
\end{align}
and we see that the length of the curve $ \mathcal{C} $ does not depend on the particular lift we choose since in the calculation tha particular $ \alpha $ we choose disappears in the product $ \left(u_{\perp}(s),u_{\perp}(s)\right) $. At this stage, some quick remarks are needed:
\begin{enumerate}
	\item The practical way to evaluate the length of a curve $ \mathcal{C} $ is to chose a particular lift, the one that is most suitable for, plug it into formula Eq.\eqref{eq:3.15} and evaluate directly. Since, therefore, it is a quantity independent from gauge transformation, it is indeed a quantity defined in the ray space $ \mathcal{R} $.
	\item The quantity $ \mathcal{L}\left[\mathcal{C}\right] $ is reparametrization invariant: this is clear in view of formula \eqref{eq:3.15} since the integrand is \textit{homogeneous of degree 1 in velocity}. This property is stated saying that $ \mathcal{L}\left[\mathcal{C}\right] $ is a geometrical object. 
\end{enumerate}
Given then the functional \eqref{eq:3.15}, using calculus of variations we can derive an equation for an extremum of functional length, in particular a minimum, that we will call \textit{geodesic}.
\begin{enumerate}[i]
	\item Suppose we take two point in ray space, let's say $ \rho_1=\rho(\psi_1^0),\rho_2=\rho(\psi_2^0) $ and suppose they're not orthogonal, in the sense that $ \Tr(\rho_1\rho_2)>0, $ i.e., $ \abs{\left(\psi_1,\psi_2\right)}\neq 0 $, then exists an unique geodesic $ \mathcal{C}_0 $ connecting $ \rho_1 $ to $ \rho_2 $ in the ray space.
	\item For our purpose is then useful to define $ \psi_1\in\pi^{-1}(\rho_1), \psi_2\in\pi^{-1}(\rho_2) $ in such a way that their inner product is real and positive definite, 
	\begin{equation}
	\left(\psi_1,\psi_2 \right)=\cos\theta, \qquad 0\leq \theta\leq\pi(\alpha)
	\end{equation}
	In general the inner product between two complex quantities gives a complex quantity. If they're real and positive definite, they're said to be in phase one each other in the Pancharatnam sense. We will see later on why this name is given. Pay attention that the concept of "Pancharatnam phase" is applied only to Hilbert space vectors and not to ray space points.
	\item If we consider the geodesic $ \mathcal{C}_0 $ from $ \rho_1 $ to $ \rho_2 $ and then we take the lift
	\begin{gather}
	\mathcal{C}_0=\left\{\psi_0(s)=\psi_1\cos(s)+(\psi_2-\psi_1\cos\theta)\frac{\sin s}{\sin\theta}\mid 0\leq s\leq\theta \right\}\subset\B\\
	\dot{\psi}_0(s)=u_0(s)=-\psi_1\sin s+(\psi_2-\psi_1\cos\theta)\dfrac{\cos s}{\sin \theta}\\
	(u_0(s),u_0(s))=1, \qquad (\psi_0(s),u_0(s))=0,\\
	u_{\perp}(s)=u_0(s),\qquad \pi\left[\mathcal{C}_0 \right]=\mathcal{C}_0\\
	\mathcal{L}\left[\pi\left[\mathcal{C}_0 \right] \right]=\theta
	\end{gather}
	in particular a parameter $ s $ is chosen in the 'affine parametrization' spirit
\end{enumerate}
In principle, no upper bounds are imposed on the value of $ \mathcal{L}\left[\mathcal{C}\right] $; it could be even infinite.

Now we want to introduce another geometrical object: we don't have time for exhaustively talk about forms, but we want at least to give to the reader the formal definition.
\begin{equation}
\mathcal{A}=\text{one-form on $ \mathcal{B} $} = -i\psi^{\dagger}\diff \psi
\end{equation}
On every $ \mathcal{C}\subset\B $ (recalling equation \eqref{eq:3.12})
\begin{equation}
\-i\int_{s_1}^{s_2}\left(\psi(s),\frac{d}{ds}\psi(s)\right)\diff s=\Im\int_{s_1}^{s_2}\left(\psi(s),u(s)\right)\diff s
\end{equation}
and the last integral does not depend on how we parametrize it; it's a geometric object.
In the language og differential geometry it is 
\begin{equation}
\int_{\mathcal{C}_0}\mathcal{A}
\end{equation}
and its integral along any curve is the dynamical phase.

% -----------------------------------------------------
% At this stage a set of examples have been given 
% but not trascripted
% -----------------------------------------------------

\section{Samuel and Bhandari generalization (1988)}
The previous mathematical section was aimed at the explanation of \cite{samuel1988general}

The main result \cite{aharonov1987phase} achieved was that only a cyclic condition was needed for the presence of a geometric phase. Samuel and Bhandari showed that even the cyclic condition is not needed.
%\textbf{cyclic condition is also not needed for achieving a geometric phase}
Let us suppose to have a generic curve in $ \mathcal{R} $ whose only requirement is to be a geodesic 
\begin{align}
C_0= \text{geodesic in $ \mathcal{R} $}&\Rightarrow \mathcal{C}_0=\text{special lift of $ \mathcal{C}_0 $ to $ \B $}\\
&\Rightarrow \mathcal{C}=\text{general lift of $ \mathcal{C}_0 $ to $ \B $}
\end{align}
The result they achieved is that the phase gained along a path is equal to the integral of the one-form defined above: in particular, the result is independent of the lifts chosen.
\begin{equation}
\int_{\mathcal{C}_0}\mathcal{A}=\Im\int_{0}^{\theta}(\psi(s),u(s)\diff S=\arg(\psi(0),\psi(\theta))
\end{equation}
So how's they did: take the hermitian time dependent Hamiltonian $ \Ham(t) $
\begin{equation}
i\hbar \ddt \psi(t)=\Ham(t)\psi(t)
\end{equation}
and now consider the set of points 
\begin{equation}
\left\{0\leq t\leq T\colon\psi(t) \text{ is a solution} \right\}\equiv \mathcal{C}\subset \B
\end{equation}
We are now defining a path that start from $ \psi(0) $ and ends at $ \psi(T) $ for an arbitrary $ T $. From $ \psi(T) $ we want now to come back to $ \psi(0) $ following the unique geodesic connecting those two points. The final path will be then given by
\begin{equation}
\mathcal{C}'=\left\{\text{\Sch evolution from $ \psi(0) $ to $ \psi(T) $} \right\} \cup \left\{\text{Any geodesic from $ \psi(T) $ to $ \psi(0) $} \right\}
\end{equation}
That is, the path $ \mathcal{C}' $ is a closed loop in the Hilbert space. The main result is then that the geometric phase of the wavefunction is given by
\begin{equation}
\phi_{geom}\left[\text{Energetic \Sch equation} \right]=\oint_{\mathcal{C}'}\mathcal{A}.
\end{equation}
This is the tool in order to relax the assumption of the cyclic Hamiltonian.
%\input{Chapters}
%\input{Chapters}
%\input{Chapters}
%\input{Chapters}
%\nocite{*}
%%%%%%%%%%%%%%%%%%%%%%%%%%%%%%%%%%%%%%%%%%%%%%%%%%%%%%%%%%%%%%%%%%%%%%%%%%%%%%
% FINE CORPO DEL DIARIO....
%%%%%%%%%%%%%%%%%%%%%%%%%%%%%%%%%%%%%%%%%%%%%%%%%%%%%%%%%%%%%%%%%%%%%%%%%%%%%%
\backmatter

\addcontentsline{toc}{chapter}{Bibliography}
%\bibliographystyle{plainnat}
\bibliography{bib}


\end{document}
