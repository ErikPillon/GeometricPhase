% !TeX encoding = UTF-8
% !TeX spellcheck = en_US
\chapter{Aharonov-Anandan and Samuel-Bhandari generalizations}
\section{Aharonov-Anandan generalization (1987)}
The first relaxation was taken in \cite{aharonov1987phase} where the authors considered the Adiabatic Hypothesis: they showed that the latter is not needed in order to get a geometric effect. 

Let us take a $ t $-dependent Hamiltonian
\begin{equation}
i\hbar\ddt\psi(t)=\Ham(t)\psi(t)
\end{equation}
and assume that we found a solution for the problem that is cyclic in the sense given in the previous chapter, that is,
\begin{equation}
\psi(T)=e^{i\psi_{tot}}\psi(0)
\label{eq:3.1}
\end{equation}
No assumptions on the cyclicity of the Hamiltonian is assumed in \eqref{eq:3.1}, nor no assumption on the adiabaticity of the Hamiltonian has been made neither and we also don't need the fact that $ \Ham(T)=\Ham(0) $, but if we have a periodicity on the solution we can associate to the \wf a new concept of geometric phase\footnote{from now on we'll distinguish the Berry phase from the present geometric phase since the concept we're dealing right now is quite different form the initial idea given by Berry: the concept of cyclicity and adiabaticity are now not needed.}: 
\begin{equation}
\phi_{geom}=\phi_{tot}-\phi_{dyn}
\end{equation}
where
\begin{equation}
\phi_{dyn}=-\dfrac{i}{\hbar}\int_{0}^{T}\Bra{\psi(t)}\Ham(t)\Ket{\psi(t)}\diff t
\label{eq:phi_dyn}
\end{equation}
Before exploit extensively the result achieved by Aharonov and Anandan, we can summarize them briefly:
\begin{enumerate}
	\item \textbf{Clear definition of geometric phase}: under the condition of cyclicity of the solution $ \psi(t) $, it is possible to derive the quantity $ \phi_{tot} $. Then, given the prescription for the definition of a dynamical phase, Eq.~\eqref{eq:phi_dyn}, is possible to derive the geometric phase understood as a difference between two terms.
	In particular, in the case of an adiabatic condition for the Hamiltonian, the calculation reduces to the ones already performed by Berry.
	\item \textbf{The geometric phase depends only on the projection on the ray space}: it is not anymore important the \wf itself but it is only important the movement of the trace of the solution projected on the space of density matrices. Being a pure state, well defined, single valued unitary vector in Hilbert space for each time, $ \psi(t) $ can be uniquely determined by the corresponding 1 dimensional projection operator, or, as we will call it, \emph{a point in the ray space}. In a certain sense, Berry already stated this result, but the A-A were able to state this result in a more general perspective, stating that the geometric phase lives in the ray space.
	\item \textbf{The parameter space concept is not needed to arrive to tha concept og geometric phase}: we're not required to know the Hamiltonian acquire its dependence on the \rp vector. 
\end{enumerate}
\section{First Mathematical Interlude}
\begin{equation}
\B=\left\{\psi\in\Hilbert\colon \Braket{\psi,\psi}=1\right\}\subset\Hilbert
\end{equation}
We notice immediately that $ \B $ is not a linear vector space. We define therefore the transformation
\begin{equation}
\psi\in\B\Rightarrow \psi'=e^{i\alpha}\psi\in\B,\qquad 0\leq\alpha\leq 2\pi
\end{equation}
The idea is that two vectors that differ only by a phase factor should be regarded as equivalent. We can also define now the equivalence class by fixing $ \psi $ and letting $ \alpha $ varying:
\begin{equation}
\mathcal{R}:=\left\{\rho(\psi)=\Ket{\psi}\Bra{\psi} \text{ or } \psi^{\dagger}\psi | \psi\in\B \right\}.
\end{equation}
We will call $ \mathcal{R} $ \textit{real space}, while $ \rho $ is the well known projective operator. We notice that if we denote by $ N $ the dimension of the Hilbert space $ \Hilbert $, then we have that $ \dim(\mathcal{R}) $ has $ 2(N-1) $ real dimensions (or equivalently $ N-1 $ complex dimensions).
We define as well the inverse of the projection operator, the operator $ \pi $, in the following way:
\begin{equation}
\pi\colon \B\to\mathcal{R} \qquad \colon \qquad \psi\in\B\rightarrow \rho(\psi)=\psi^{\dagger}\psi\in\mathcal{R}
\end{equation}
and goes into $ CP^{N-1} $.

Points in the real space are in $ 1-1 $ correspondence with the pure real state of the system. The mathematical description is
\begin{equation}
\rho(\psi)\in\mathcal{R}\to\pi^{-1}(\rho(\psi)):=\left\{\psi'=e^{i\alpha}\psi\in\B|\psi \text{ fixed }, 0\leq\alpha\leq 2\pi \right\}\subset \B
\end{equation}

We now consider curves on the space $ \B $. Given a curve 
\begin{equation}
\mathcal{C}:=\left\{\psi(s)\in\B |s_1\leq s\leq s_2 \right\}\subset \B
\end{equation}
we consider its projection into the real space $ \mathcal{R} $ through $ \pi $ into the set 
\begin{equation}
C=\pi\left[\mathcal{C}\right]=\left\{\rho(\psi(s))=\psi(s)\psi^{\dagger}(s)\in\mathcal{R}|s_1\leq s\leq s_2 \right\}.
\end{equation}
So for our purposes the most general curve will be 
\begin{equation}
\mathcal{C}'=\left\{\psi'(s)=e^{i\alpha(s)}\psi(s)|\psi(s)\in\mathcal{C}, s_1\leq s\leq s_2 \right\},
\end{equation}
that denotes all the possible lifts of the original projected curve. Now comes natural to define the tangent vector 
\begin{align}
	u(s)&=\dfrac{d}{ds}\psi(s)=\dot{\psi}(s)\\
	(\psi(s),\dot{\psi}(s))&=\Im(\psi(s),u(s))\label{eq:3.12}\\
	u'(s)&=e^{i\alpha(s)}(u(s)+i\dot{\alpha}(s)\psi(s))\\
	u_{\perp}(s)&=u(s)-\psi(s)(\psi(s),u(s))=u_{\perp}'(s)=e^{i\alpha(s)}u_{\perp}(s)\\
	\mathcal{L}\left[\mathcal{C}\right]&=\int_{s_1}^{s_2}\left(u_{\perp}(s),u_{\perp}(s)\right)^{1/2}\cdot\diff s \\
	&=\int_{s_1}^{s_2}\left\{\left(\dot{\psi}(s),\dot{\psi}(s)\right)-\left(\psi(s),\dot{\psi}(s)\right)-\left(\dot{\psi}(s),\psi(s)\right) \right\}^{1/2} \diff s
\end{align}

\begin{enumerate}[i]
	\item $ \rho_1=\rho(\psi_1),\rho_2=\rho(\psi_2) $; then $ \Tr(\rho_1\rho_2)>0, $ i.e., $ \abs{\left(\psi_1,\psi_2\right)}\neq 0 $, then exists an unique geodesic $ \mathcal{C}_0 $
	\item $ \psi_1\in\pi^{-1}(\rho_1), \psi_2\in\pi^{-1}(\rho_2) $
	\begin{equation}
	\left(\psi_1,\psi_2 \right)=\cos\theta, \qquad 0\leq \theta\leq\pi(\alpha)
	\end{equation}
	In general the inner product between two complex quantities gives a complex quantity. If they're real and positive definite, they're said to be in phase one each other in the Plancherel sense.
	\item If we consider the geodesic $ \mathcal{C}_0 $ from $ \rho_1 $ to $ \rho_2 $ and then we take the lift
	\begin{gather}
	\mathcal{C}_0=\left\{\psi_0(s)=\psi_1\cos(s)+(\psi_2-\psi_1\cos\theta)\frac{\sin s}{\sin\theta}\mid 0\leq s\leq\theta \right\}\subset\B\\
	\dot{\psi}_0(s)=u_0(s)=-\psi_1\sin s+(\psi_2-\psi_1\cos\theta)\dfrac{\cos s}{\sin \theta}\\
	(u_0(s),u_0(s))=1, \qquad (\psi_0(s),u_0(s))=0,\\
	u_{\perp}(s)=u_0(s),\qquad \pi\left[\mathcal{C}_0 \right]=\mathcal{C}_0\\
	\mathcal{L}\left[\pi\left[\mathcal{C}_0 \right] \right]=\theta
	\end{gather}
\end{enumerate}

Now we want to introsuce another geometrical object: 
\begin{equation}
\mathcal{A}=\text{one-form on $ \mathcal{B} $} = -i\psi^{\dagger}\diff \psi
\end{equation}
On every $ \mathcal{C}\subset\B $ (recalling equation \eqref{eq:3.12})
\begin{equation}
\-i\int_{s_1}^{s_2}\left(\psi(s),\frac{d}{ds}\psi(s)\right)\diff s=\Im\int_{s_1}^{s_2}\left(\psi(s),u(s)\right)\diff s
\end{equation}
and the last integral does not depend on how we parametrize it; it's a geometric object.
In the language og differential geometry it is 
\begin{equation}
\int_{\mathcal{C}_0}\mathcal{A}
\end{equation}
and its integral along any curve is the dynamical phase.

\section{S-R generalization (1988)}
\textbf{cyclic condition is also not needed for achieving a geometric phase}

\begin{align}
\mathcal{C_0}= \text{geodesic in $ \mathcal{R} $}&\Rightarrow \mathcal{C_0}=\text{special lift of $ \mathcal{C}_0 $ to $ \B $}\\
&\Rightarrow \mathcal{C}=\text{general lift of $ \mathcal{C}_0 $ to $ \B $}
\end{align}

\begin{equation}
\int_{\mathcal{C}_0}\mathcal{A}=\Im\int_{0}^{\theta}(\psi(s),u(s)\diff S=\arg(\psi(0),\psi(\theta))
\end{equation}
So how's they did: take the hermitian Hamiltonian $ \Ham $
\begin{equation}
i\hbar \ddt \psi(t)=\Ham(t)\psi(t)
\end{equation}
and now consider the set of points 
\begin{equation}
\left\{0\leq t\leq T\colon\psi(t) \text{ is a solution} \right\}\equiv \mathcal{C}_0\subset \B
\end{equation}

\begin{equation}
\phi_{geom}\left[\text{Energetic \Sch equation} \right]=\oint_{\mathcal{C}_0}\mathcal{A}.
\end{equation}
This is the tool in order to relax the assumption of the cyclic Hamiltonian.